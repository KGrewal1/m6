
\documentclass[a4paper, 12pt]{article}
\usepackage[]{graphicx}
\usepackage[]{color}

\makeatletter
\def\maxwidth{ %
	\ifdim\Gin@nat@width>\linewidth
	\linewidth
	\else
	\Gin@nat@width
	\fi
}
\makeatother

\definecolor{fgcolor}{rgb}{0.345, 0.345, 0.345}
\newcommand{\hlnum}[1]{\textcolor[rgb]{0.686,0.059,0.569}{#1}}%
\newcommand{\hlstr}[1]{\textcolor[rgb]{0.192,0.494,0.8}{#1}}%
\newcommand{\hlcom}[1]{\textcolor[rgb]{0.678,0.584,0.686}{\textit{#1}}}%
\newcommand{\hlopt}[1]{\textcolor[rgb]{0,0,0}{#1}}%6
\newcommand{\hlstd}[1]{\textcolor[rgb]{0.345,0.345,0.345}{#1}}%
\newcommand{\hlkwa}[1]{\textcolor[rgb]{0.161,0.373,0.58}{\textbf{#1}}}%
\newcommand{\hlkwb}[1]{\textcolor[rgb]{0.69,0.353,0.396}{#1}}%
\newcommand{\hlkwc}[1]{\textcolor[rgb]{0.333,0.667,0.333}{#1}}%
\newcommand{\hlkwd}[1]{\textcolor[rgb]{0.737,0.353,0.396}{\textbf{#1}}}%
\let\hlipl\hlkwb

\usepackage{soul}
\usepackage{framed}

\makeatletter
\newenvironment{kframe}{%
	\def\at@end@of@kframe{}%
	\ifinner\ifhmode%
	\def\at@end@of@kframe{\end{minipage}}%
\begin{minipage}{\columnwidth}%
	\fi\fi%
	\def\FrameCommand##1{\hskip\@totalleftmargin \hskip-\fboxsep
		\colorbox{shadecolor}{##1}\hskip-\fboxsep
		% There is no \\@totalrightmargin, so:
		\hskip-\linewidth \hskip-\@totalleftmargin \hskip\columnwidth}%
	\MakeFramed {\advance\hsize-\width
		\@totalleftmargin\z@ \linewidth\hsize
		\@setminipage}}%
{\par\unskip\endMakeFramed%
	\at@end@of@kframe}
\makeatother

\definecolor{shadecolor}{rgb}{.97, .97, .97}
\definecolor{messagecolor}{rgb}{0, 0, 0}
\definecolor{warningcolor}{rgb}{1, 0, 1}
\definecolor{errorcolor}{rgb}{1, 0, 0}

%\usepackage[ascii]{inputenc}
\usepackage{amsmath}
\usepackage{amssymb,amsfonts,textcomp}
\usepackage[T1]{fontenc}
\usepackage[german,french,english]{babel}
\usepackage{array}
\usepackage{hhline}
\usepackage{hyperref}
\usepackage{alltt}
\usepackage[version=4]{mhchem}
\usepackage[margin=0.75in]{geometry}
%\usepackage[format=plain, labelfont={bf,it}, textfont=it, margin=1in]{caption}
\usepackage{subcaption}
\usepackage{graphicx}
%\usepackage{subfig}
\usepackage[format=plain, labelfont={bf}, textfont=it, margin=1in]{caption}
\graphicspath{ {./images/} }
%\usepackage{sidecap}
%\usepackage{svg}
\usepackage{lastpage}
\usepackage{fancyhdr}
%\usepackage{titlesec} %For section title spacing
\usepackage{enumitem} % for the hanging description
\usepackage{color}
\usepackage[english]{babel}
\usepackage{bm}
\usepackage{wasysym}
\usepackage{natbib}

\usepackage{comment}
\usepackage{pgfplots}
\usepackage{multirow}
\usepackage{textgreek}
% Bibliography
\usepackage[utf8]{inputenc}
%\usepackage[backend=bibtex,bibencoding=ascii]{biblatex}
%\usepackage[backend=biber]{biblatex}
%\addbibresource{SpecBiblo.bib}
%\usepackage{csquotes} % It is recommended to use csquotes when using babel with biblatex
\usepackage{pdflscape}
\usepackage{braket}
\usepackage{siunitx}
%\usepackage{esint}




%%%%%%%%%%%%%%%%%%%%%%%%%%%%%%%%%%%%%%%%%%%%%%%%%%%%%%%%%%%%%%%%%%%%%%%%%%%%%%%%%%%%%%%%%%%
%Date, Title, Subject, No
%%%%%%%%%%%%%%%%%%%%%%%%%%%%%%%%%%%%%%%%%%%%%%%%%%%%%%%%%%%%%%%%%%%%%%%%%%%%%%%%%%%%%%%%%%%
%Set Date
\usepackage{pgfcalendar}
\usepackage{datetime2}
\usepackage{datetime2-calc}
\DTMsavedate{MyDate}{2024-02-06}\DTMmakeglobal{MyDate}
%Set Subject
\newcommand{\Subject}{M6}
%Set Type
\newcommand{\Type}{Supervision}
%Set Number
\newcommand{\Iteration}{1}
%For out of uni work
\newcommand{\FullAuthor}{Kirpal Grewal\\\href{mailto:kg488@cam.ac.uk}{\color{black}kg488@cam.ac.uk}\\St Catharine's College\\University of Cambridge}
%For practicals,out of collge work
\newcommand{\LongAuthor}{Kirpal Grewal\\\href{mailto:kg488@cam.ac.uk}{\color{black}kg488}\\St Catharine's College}
%For supervisions
\newcommand{\ShortAuthor}{Kirpal Grewal}
%%%%%%%%%%%%%%%%%%%%%%%%%%%%%%%%%%%%%%%%%%%%%%%%%%%%%%%%%%%%%%%%%%%%%%%%%%%%%%%%%%%%%%%%%%%

%%%%%%%%%%%%%%%%%%%%%%%%%%%%%%%%%%%%%%%%%%%%%%%%%%%%%%%%%%%%%%%%%%%%%%%%%%%%%%%%%%%%%%%%%%%
%Full Title: change here if not for Supo work/practicals
\newcommand{\MyTitle}{\Subject{ }\Type{ }\Iteration{ }}




% Set up references for pdf
\hypersetup{
	colorlinks=true,
	linkcolor=black,
	citecolor=black,
	filecolor=blue,
	urlcolor=blue,
	pdftitle={\MyTitle},
	pdfauthor={Kirpal Grewal},
	pdfsubject={Natural Science, Homework},
	pdfkeywords={Chemistry, Natural Sciences}
}

\hyphenation{Marvin-Sketch}
\pagestyle{fancyplain}
\rhead{}
\lhead{}
\lhead{Kirpal Grewal, \href{mailto:kg488@cam.ac.uk}{\color{black}kg488@cam.ac.uk}}
\rhead{\MyTitle}
\lfoot{\DTMfetchday{MyDate}{ }\DTMenglishmonthname{\DTMfetchmonth{MyDate}}{ }\DTMfetchyear{MyDate}}

\rfoot{Page \thepage\ of \pageref{LastPage}}
\cfoot{ }
\renewcommand{\headrulewidth}{0.4pt}
\renewcommand{\footrulewidth}{0.4pt}


% footnotes configuration
\makeatletter
\renewcommand\thefootnote{\arabic{footnote}}
\makeatother
% Text styles
\newcommand\textstyleFootnoteReference[1]{\textsuperscript{#1}}
% Outline numbering
\setcounter{secnumdepth}{0}


\newcommand\liststyleLFOiii{%
	\renewcommand\labelitemi{{\textbullet}}
	\renewcommand\labelitemii{o}
	\renewcommand\labelitemiii{${\blacksquare}$}
	\renewcommand\labelitemiv{{\textbullet}}
}
\newcommand\liststyleLFOii{%
	\renewcommand\labelitemi{{\textbullet}}
	\renewcommand\labelitemii{o}
	\renewcommand\labelitemiii{${\blacksquare}$}
	\renewcommand\labelitemiv{{\textbullet}}
}

\newcommand{\snt}[1]{\textcolor{magenta}{#1}} % edited by Sergei
\newcommand{\crsid}[1]{\textcolor{red}{#1}} % edited by crsid
\newcommand{\sntdel}[1]{\textcolor{green}{\sout{#1}}} % deleted by Sergei

\newcommand{\problem}[1]{\subsection*{#1}
	\setcounter{equation}{0}}
\newcommand{\question}[1]{\problem{Q. #1}}


\DeclareMathOperator{\sech}{sech}
\DeclareMathOperator{\csch}{csch}
\DeclareMathOperator{\arcsec}{arcsec}
\DeclareMathOperator{\arccot}{arccot}
\DeclareMathOperator{\arccsc}{arccsc}
\DeclareMathOperator{\arcosh}{arcosh}
\DeclareMathOperator{\arsinh}{arsinh}
\DeclareMathOperator{\artanh}{artanh}
\DeclareMathOperator{\arsech}{arsech}
\DeclareMathOperator{\arcsch}{arcsch}
\DeclareMathOperator{\arcoth}{arcoth}
\DeclareMathOperator{\sgn}{sgn}
\DeclareMathOperator{\adj}{adj}
\DeclareMathOperator{\drv}{d}
\DeclareMathOperator{\trace}{tr}
\DeclareMathOperator{\Prob}{P}
\newcommand*{\Perm}[2]{{}^{#1}\!P_{#2}}%
\newcommand*{\Comb}[2]{{}^{#1}C_{#2}}%




\title{Natural Sciences Tripos \MyTitle}
\author{\ShortAuthor}

\date{{\DTMusedate{MyDate}}}

\begin{document}
\maketitle

\question{1.1}
\begin{flalign*}
	\bm{P}&=\sum_{i=1}^{N}\bm{p}_{i}\\
	\frac{d \bm{p}_{i}}{d t} &=-\sum_{j\neq i}^{N}\frac{du(r_{ij})}{d_{r}}\frac{\bm{r}_{i}-\bm{r}_{j}}{r_{ij}}\\
	\intertext{ie the sum of all pairwise forces acting on the molecule}
	\frac{d \bm{P}}{d t}&= \frac{d \bm{p}_{i}}{d t}\\
	\intertext{consider particles $m$ and $n$ and their force on each other}
	\frac{d \bm{p}_{m}}{d t} &=-\frac{du(r_{nm})}{d_{r}}\frac{\bm{r}_{n}-\bm{r}_{m}}{r_{nm}}\\
	\frac{d \bm{p}_{m}}{d t} &=-\frac{du(r_{mn})}{d_{r}}\frac{\bm{r}_{m}-\bm{r}_{n}}{r_{mn}}\\
	\intertext{It can be seen that these have the same magnitude but oopposite direction and cancel each other out in the contribution to the overall potential: as we have assumed nothing about $n$ and $m$ this cancellation holds for all pairs of particles and the overall momentum of the system is time invariant}
	\intertext{This follows from Noether's theorem as the system is invariant upon translation, the symmetry corresponding to conservation of the momentum of the system}
\end{flalign*}
\clearpage
\question{1.2}
For the leapfrog algorithm:
\begin{flalign*}
	\bm{v}_{i}(t+\delta t /2) &=\bm{v}_{i}(t-\delta t/2)+\frac{\delta t}{m_{i}}\bm{f}_{i}(t)\\
	\bm{r}_{i}(t+\delta t) &= \bm{r}_{i}(t)+\bm{v}_{i}(t+\delta t/2)\delta t
\end{flalign*}

\begin{flalign*}
	\bm{r}_{i}(t+\delta t) &= \bm{r}_{i}(t)+\bm{v}_{i}(t+\delta t/2)\delta t\\
	\bm{r}_{i}(t) &= \bm{r}_{i}(t-\delta t)+\bm{v}_{i}(t-\delta t/2)\delta t\\
	\bm{r}_{i}(t+\delta t)-\bm{r}_{i}(t) &= \bm{r}_{i}(t)- \bm{r}_{i}(t-\delta t)+(\bm{v}_{i}(t+\delta t/2)-\bm{v}_{i}(t-\delta t/2))\delta t\\
	\intertext{And substituting in the velocity change from the algorithm}
	\bm{r}_{i}(t+\delta t)-\bm{r}_{i}(t) &= \bm{r}_{i}(t)- \bm{r}_{i}(t-\delta t)+\frac{\delta t^{2}}{m_{i}}\bm{f}_{i}(t)\\
	\intertext{The velocity Verlet}
\end{flalign*}

\question{1.3}
a
\begin{flalign*}
	\braket{A(t)}&=\lim_{\tau\to \infty}\frac{1}{\tau}\int_{0}^{\tau}\cos( \omega t) d t\\
	&=\lim_{\tau\to \infty}\frac{1}{\omega\tau }\sin \omega \tau \\
	&=0\\
	\braket{A(t)^{2}}&=\lim_{\tau\to \infty}\frac{1}{\tau}\int_{0}^{\tau}\cos( \omega t)^{2} d t\\
	&=\lim_{\tau\to \infty}\frac{1}{4\omega\tau }\left(2\tau +\sin(2\tau\omega)\right) \\
	&=\frac{1}{2}\\
	C_{AA}(t-t')&=\braket{\cos (\omega t))(\cos (\omega t')}\\
	C_{AA}(\tau)&=\frac{1}{t}\int_{0}^{\infty}\cos (\omega t)(\cos (\omega (t-\tau)))dt\\
	&=\frac{1}{t}\left[\frac{1}{2}t\cos(\omega \tau)-\frac{\sin(\omega(\tau - 2t))}{4\omega}\right]\\
	\intertext{Again the term in $\sin$ vanishes in the limit leaving}
	&=\frac{1}{2}\cos(\omega \tau)\\
\end{flalign*}
b
\begin{flalign*}
	\intertext{For any superposition of harmonic modes it is clear the time average will be 0}
	C_{AA}(\tau)&=\frac{1}{t}\int_{0}^{\infty}(a_{1}\cos (\omega_{1} t)+a_{2}\cos (\omega_{2} t))(a_{1}\cos (\omega_{1} (t-\tau))+a_{2}\cos (\omega_{2} (t-\tau)))dt\\
	&=\frac{a_{1}^{2}}{2}\cos(\omega_{1} \tau)+\frac{a_{2}^{2}}{2}\cos(\omega_{2} \tau)+\frac{a_{1}a_{2}}{t}\int_{0}^{\infty}(\cos (\omega_{1} t)\cos (\omega_{2} (t-\tau)))+(\cos (\omega_{1} (t-\tau))\cos (\omega_{2} t))dt\\
	\intertext{Using the orthogonality of sinusoidal functions of different frequencies, this integral is 0 in the limit}
	&=\frac{a_{1}^{2}}{2}\cos(\omega_{1} \tau)+\frac{a_{2}^{2}}{2}\cos(\omega_{2} \tau)\\
\end{flalign*}
\end{document}
